\documentclass[../generics]{subfiles}

\begin{document}

\chapter{Type Substitution}\label{notation summary}

We've been using the \index{type substitution}type substitution algebra for describing identities among types, conformances and substitution maps. The formalism was introduced over the course of Chapters \ref{substmaps}, \ref{conformances}, and \ref{conformance paths}.

The entities being operated upon:
\begin{itemize}
\item \IndexSet{proto}{\ProtoObj}$\ProtoObj$ is the set of all \index{protocol declaration}protocols.
\item \IndexSet{type}{\TypeObj{G}}$\TypeObj{G}$ is the set of all \index{interface type}interface types valid in the \index{generic signature}generic signature $G$ (see \SecRef{derived req}).
\item \IndexSet{sub}{\SubMapObj{G}{H}}$\SubMapObj{G}{H}$ is the set of all \index{substitution map}substitution maps with \index{input generic signature}input generic signature $G$ and \index{output generic signature}output generic signature $H$.
\item \IndexSet{conf}{\ConfObj{G}}$\ConfObj{G}$ is the set of all \index{conformance}conformances with output generic signature $G$.
\item \IndexSet{confp}{\ConfPObj{P}{G}}$\ConfPObj{P}{G}$ is the set of all conformances to protocol \texttt{P} with output generic signature $G$.
\item \IndexSet{assoctype}{\AssocTypeObj{P}}$\AssocTypeObj{P}$ is the set of all \index{associated type declaration}associated type declarations in protocol $\texttt{P}$.
\item \IndexSet{assocconf}{\AssocConfObj{P}}$\AssocConfObj{P}$ is the set of all \index{associated conformance requirement}associated conformance requirements in protocol $\texttt{P}$.
\item \IndexSet{req}{\ReqObj{G}}$\ReqObj{G}$ is the set of all \index{requirement}requirements containing interface types of $G$.
\end{itemize}
A composition operation is defined on various kinds of entities, written as \index{$\otimes$}$A \otimes B$. If a juxtaposition of three entities can be evaluated in two different ways, the result is always the same, so the composition operator is \emph{associative}: $A\otimes (B\otimes C) = (A\otimes B) \otimes C$.

Substitution maps act on the right of types, conformances, other substitution maps, and requirements. This gives us \index{type substitution}type substitution (\SecRef{substmaps}), \index{conformance substitution map}conformance substitution (\ChapRef{conformances}), \index{substitution map composition}substitution map composition (\SecRef{submapcomposition}), and \index{substituted requirement}requirement substitution (\SecRef{checking generic arguments}). This action produces a new entity of the same kind as the original:
\begin{gather*}
\TypeObj{G}\otimes\SubMapObj{G}{H}\longrightarrow\TypeObj{H}\\
\ConfObj{G}\otimes\SubMapObj{G}{H}\longrightarrow\ConfObj{H}\\
\SubMapObj{F}{G}\otimes\SubMapObj{G}{H}\longrightarrow\SubMapObj{F}{H}\\
\ReqObj{G}\otimes\SubMapObj{G}{H}\rightarrow\ReqObj{H}
\end{gather*}

Protocols act on the left of types (\index{global conformance lookup}global conformance lookup, \SecRef{conformance lookup}):
\[\ProtoObj\otimes\TypeObj{G}\longrightarrow\ConfObj{G}\]
Associated type declarations act on the left of conformances (\index{type witness}type witness projection, \SecRef{type witnesses}):
\[\AssocTypeObj{P}\otimes\ConfPObj{P}{G}\longrightarrow\TypeObj{G}\]
Associated conformance requirements act on the left of conformances (\index{associated conformance projection}associated conformance projection, \SecRef{associated conformances}):
\[\AssocConfObj{P}\otimes\ConfPObj{P}{G}\longrightarrow\ConfPObj{P}{G}\]

\paragraph{Substitution maps.}
A substitution map is completely determined by its action on each generic parameter type and root abstract conformance. Type substitution with a \index{generic parameter type}generic parameter type projects the replacement type from the substitution map:
\[\ttgp{d}{i} \otimes \SubstMapC{\ldots,\,\SubstType{\ttgp{d}{i}}{Int},\,\ldots}{\ldots} := \texttt{Int}\]
Conformance substitution with a \index{root abstract conformance}root abstract conformance projects the corresponding conformance from the substitution map:
\[
\ConfReq{T}{Q} \otimes \SubstMapC{\ldots}{\ldots,\,\SubstConf{T}{G<Int>}{Q},\,\ldots} := \ConfReq{G<Int>}{Q}
\]
If $\texttt{T}\in\TypeObj{G_1}$, $\TP\in\ConfPObj{P}{G_1}$, $\Sigma_1\in\SubMapObj{G_1}{G_2}$, $\Sigma_2\in\SubMapObj{G_2}{G_3}$, $\Sigma_3\in\SubMapObj{G_3}{G_4}$, then 
\begin{gather*}
(\texttt{T}\otimes \Sigma_1) \otimes \Sigma_2=\texttt{T}\otimes (\Sigma_1\otimes \Sigma_2)\\
(\TP\otimes \Sigma_1) \otimes \Sigma_2=\TP\otimes (\Sigma_1\otimes \Sigma_2)\\
(\Sigma_1\otimes \Sigma_2) \otimes \Sigma_3=\Sigma_1\otimes(\Sigma_2\otimes \Sigma_3)
\end{gather*}

\paragraph{Conformances.} Type witness and associated conformance projection is compatible with conformance substitution. Suppose $\TP\in\ConfPObj{P}{G}$, $\AssocType{[P]A}\in\AssocTypeObj{P}$, $\AssocConf{Self.U}{Q}\in\AssocConfObj{P}$, and $\Sigma\in\SubMapObj{G}{H}$, then
\begin{gather*}
\bigl(\AssocType{[P]A}\otimes \TP\bigr)\otimes \Sigma=\AssocType{[P]A}\otimes (\TP\otimes \Sigma)\\
\bigl(\AssocConf{Self.U}{Q}\otimes \TP\bigr)\otimes \Sigma=\AssocConf{Self.U}{Q}\otimes \bigl(\TP\otimes \Sigma\bigr)
\end{gather*}
If $d$ is a \index{nominal type declaration}nominal type declaration conforming to \texttt{P}, and $G$ is the generic signature of the conformance context, then there exists a \index{normal conformance}normal conformance $\ConfReq{$\texttt{T}_d$}{P}\in\ConfPObj{P}{G}$, and
\[\pP\otimes\texttt{T}_d=\ConfReq{$\texttt{T}_d$}{P}\]
If $H$ is some other generic signature, $\Sigma\in\SubMapObj{G}{H}$, and $\texttt{T}=\texttt{T}_d\otimes \Sigma$, then there exists a \index{specialized conformance}specialized conformance $\TP\in\ConfPObj{P}{H}$, and
\[\pP\otimes\texttt{T}=\ConfReq{$\texttt{T}_d$}{P}\otimes \Sigma = \TP\]
If \texttt{T} is a \index{type parameter}type parameter of $G$ and the conformance requirement $\TP$ can be derived from $G$, then there exists an \index{abstract conformance}abstract conformance $\TP\in\ConfPObj{G}{P}$, and
\[\pP\otimes\texttt{T} = \TP\]
Global conformance lookup is compatible with type substitution:
\[(\pP\otimes\texttt{T})\otimes \Sigma = \texttt{P}\otimes(\texttt{T}\otimes \Sigma)\]

\paragraph{Abstract conformances.} The type witnesses of an abstract conformance are \index{dependent member type}dependent member types, and the associated conformances are other abstract conformances:
\begin{gather*}
\AssocType{[P]A} \otimes \TP = \texttt{T.[P]A}\\
\AssocConf{Self.U}{Q} \otimes \TP = \ConfReq{T.U}{Q}
\end{gather*}
Applying a substitution map to an abstract conformance is called \index{local conformance lookup}local conformance lookup (\SecRef{abstract conformances}). The first identity above defines substitution of dependent member types in terms of local conformance lookup:
\[\texttt{T.[P]A} \otimes \Sigma = \AssocType{[P]A} \otimes (\TP \otimes \Sigma)\]
Every abstract conformance has a unique factorization as a \index{reduced conformance path}reduced conformance path. A conformance path is a sequence of zero or more associated conformance projections applied to an root abstract conformance:
\[
\AssocConf{Self.[$\texttt{P}_{n+1}$]$\texttt{A}_n$}{$\texttt{P}_n$}
\otimes
\cdots
\otimes
\AssocConf{Self.[$\texttt{P}_2$]$\texttt{A}_1$}{$\texttt{P}_1$}
\otimes
\ConfReq{$\texttt{T}_0$}{$\texttt{P}_0$}
\]
When applied to a non-root abstract conformance, local conformance lookup first factors the abstract conformance with a conformance path. Thus, applying a substitution map to a dependent member type breaks down into this juxtaposition of entities, from \emph{right} to \emph{left}:
\begin{itemize}
\item A substitution map.
\item Then, a conformance path, consisting of a root abstract conformance, followed by zero or more associated conformance projections.
\item Finally, a type witness projection.
\end{itemize}
The substituted type of a dependent member type is computed by taking a root conformance from the substitution map, following a chain of zero or more associated conformance projections, and finally projecting the type witness.

\end{document}
