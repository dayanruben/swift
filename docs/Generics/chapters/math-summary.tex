\documentclass[../generics]{subfiles}

\begin{document}

\chapter{Mathematical Conventions}\label{math summary}

The more formal sections in this book are written in ``Euclidean style'', where the text is further subdivided into elements of the below kinds. All elements are numbered consecutively within a single chapter, except for algorithms which use a letter:
\begin{itemize}
\item A \textbf{Definition} introduces terminology or notation.
\item An \textbf{Example} shows how this terminology and notation arises in practice.
\item A \textbf{Proposition} states a logical consequence of one or more definitions.
\item A \textbf{Lemma} is an  intermediate proposition in service of proving another theorem.
\item A \textbf{Theorem} is a ``deeper'' proposition which is more profound in some sense.
\item A \textbf{Corollary} is an immediate consequence of an earlier theorem.
\item An \textbf{Algorithm} is a step-by-step description of a computable function, defined in terms of inputs and outputs.
\end{itemize}

A handful of Greek letters are used for variable names and other notation: lowercase $\alpha$~(alpha), $\beta$~(beta), $\gamma$~(gamma), $\varepsilon$~(epsilon), $\eta$~(eta), $\sigma$~(sigma), $\uptau$~(tau), $\varphi$~(phi); and uppercase $\Sigma$ (sigma). The ``='' operator says that two existing things are identical or equivalent; ``$\neq$'' means that they are not. The ``:='' operator \emph{defines} the new thing on the left as a combination of existing things on the right. The ``$\leftarrow$'' operator denotes an imperative assignment statement in an algorithm.

\paragraph{Sets.}
\IndexDefinition{set}
\IndexDefinition{natural numbers}
\IndexDefinition{integers}
\IndexDefinition{empty set}
A \emph{set} is a collection of elements without regard to order or duplicates. Sets can be finite or infinite. A finite set can be specified by listing its elements in any order, for example $\{a,\,b,\,c\}$. The empty set \index{$\varnothing$}\index{$\varnothing$!z@\igobble|seealso{empty set}}$\varnothing$ is the unique set with no elements. The set of \emph{natural numbers} \index{$\NN$}\index{$\NN$!z@\igobble|seealso{natural numbers}}$\NN:=\{0,\,1,\,2,\,\ldots\}$ is the infinite set containing zero and all of its \index{successor}successors. The set of \emph{integers} \index{$\mathbb{Z}$}\index{$\mathbb{Z}$!z@\igobble|seealso{integers}}$\mathbb{Z}:=\{\ldots,\,-2,\,-1,\,0,\,1,\,2,\,\ldots\}$ also contains the negative whole numbers.

The notation \index{$\in$}\index{$\in$!z@\igobble|seealso{set}}$x\in S$ means ``$x$ is an element of a set $S$,'' and \index{$\notin$}$x\notin S$ is its negation. Properties of sets can be defined using \emph{existential} quantification (``there exists (at least one) $x\in S$ such that $x$ has this property\ldots'') or \emph{universal} quantification (``for all $x\in S$, the following property is true of $x$\ldots'').

\IndexDefinition{subset}%
\IndexDefinition{proper subset}%
A set $X$ is a \emph{subset} of another set $Y$, written as \index{$\subseteq$}\index{$\subseteq$!z@\igobble|seealso{subset}}$X\subseteq Y\!$, if for all $x\in X$, it is also true that $x\in Y\!$. If $X\subseteq Y$ and $Y\subseteq X$, then the two sets have the same elements, so they're equal; $X=Y\!$. If $X\subseteq Y$ and $X\neq Y\!$, then $X$ is a \emph{proper} subset of $Y$, written as \index{$\subsetneq$}\index{$\subsetneq$!z@\igobble|seealso{proper subset}}$X\subsetneq Y\!$. The \IndexDefinition{union}\emph{union} \index{$\cup$}\index{$\cup$!z@\igobble|seealso{union}}$X\cup Y$ is the set of all elements belonging to either $X$ or $Y\!$. The \IndexDefinition{intersection}\emph{intersection} \index{$\cap$}\index{$\cap$!z@\igobble|seealso{intersection}}$X\cap Y$ is the set of all elements belonging to both $X$ and $Y\!$. The \IndexDefinition{set difference}\emph{difference} \index{$\setminus$}\index{$\setminus$!z@\igobble|seealso{set difference}}$X\setminus Y$ is the set of all elements of $X$ not also in $Y\!$.

\IndexDefinition{Cartesian product}
\IndexDefinition{ordered pair}
\IndexDefinition{ordered tuple}
The \emph{Cartesian product} of two sets $X$ and $Y\!$, denoted \index{$\times$}\index{$\times$!z@\igobble|seealso{Cartesian product}}$X\times Y\!$, is the set of all \emph{ordered pairs} $(x,y)$ where $x\in X$, $y\in Y\!$. Note that the ordered pair $(x,y)$ is not the same as the set $\{x,y\}$, because $(x,y)\neq(y,x)$. The Cartesian product construction generalizes to any finite number of sets, to give \emph{ordered tuples} or \emph{sequences}.

\paragraph{Functions.}
\IndexDefinition{binary operation}
\IndexDefinition{homomorphism}
\index{mapping|see{function}}
\IndexDefinition{function}
A \emph{function} (or \emph{mapping}) $f\colon X\rightarrow Y$ assigns to each $x\in X$ a unique element $f(x)\in Y$. If the sets $X$ and $Y$ are equipped with some kind of additional structure (which will be explicitly defined), then $f$ is a \emph{homomorphism} if it preserves this structure.

A function $f\colon X\times Y\rightarrow Z$ defined on the Cartesian product can be thought of as taking a pair of values $x\in X$, $y\in Y$ to an element $f(x,y)\in Z$. A \emph{binary operation} is a function named by a symbol like ``\;$\otimes$\;'', ``\;$\WL$\;'', or ``\;$\cdot$\;'' defined on the Cartesian product of two sets. The application of a binary operation is denoted by writing the symbol in between the two elements, like $x\otimes y$.

The \emph{cardinality} of a finite set $S$, denoted $|S|$, is the number of elements in $S$. The notation $|x|$ sometimes denotes certain other functions taking values in $\NN$, such as the length of a sequence; this will always be explicitly defined when needed.

\section*{More}

\begin{itemize}
\item Formal systems (\SecRef{derived req}).
\item Equivalence relations (\SecRef{valid type params}, \SecRef{rewrite graph}).
\item Partial and linear orders (\SecRef{reduced types}, \SecRef{rewritesystemintro}, \SecRef{building terms}).
\item Category theory (\SecRef{submapcomposition}).
\item Boolean satisfiability (\SecRef{associated type inference}).
\item Directed graphs (\SecRef{type parameter graph}, \SecRef{finding conformance paths}, \SecRef{recursive conformances}, \SecRef{protocol component}).
\item Proof by induction (\SecRef{generic signature validity}).
\item Computability theory (\SecRef{tag systems}, \SecRef{word problem}).
\item Finitely-presented monoids and string rewriting (\ChapRef{monoids}, \ChapRef{completion}).
\end{itemize}

\end{document}
